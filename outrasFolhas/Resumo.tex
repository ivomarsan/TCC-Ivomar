\vfill
\begin{center}
{\textbf{RESUMO}\\}
\end{center}
\noindent

Atualmente, é perceptível a crescente utilização e difusão dos serviços existentes na rede mundial de computadores, também conhecida pela sigla WWW (em inglês \textit{World Wide Web}) ou simplesmente \textit{Web}, que é um sistema de documentos interligados e que são aplicados na Internet. Sua arquitetura é dividida principalmente em duas partes interconectadas, chamadas de cliente e servidor. Neste modelo, as responsabilidades são repartidas e juntas garantem a execução de serviços e sistemas de redes. O lado servidor, como o nome explicita, é responsável por servir seus clientes, disponibilizando o serviço para responder a uma demanda do cliente. Por outro lado, o cliente requisita o serviço, ou seja, inicia a interação a partir de uma requisição do usuário, através de um navegador (\textit{browser}). Desta forma, o usuário pode realizar requisições de conteúdo dinâmico com o servidor, através de diversos protocolos de comunicação, sendo o HTTP (\textit{Hypertext Transfer Protocol}) o mais comum. Além disso, as tecnologias empregadas para o perfeito funcionamento de uma aplicação \textit{Web} também são subdivididas. É comum observar que, geralmente, toda interação visual está localizado no cliente. Já no servidor, localiza-se todo o serviço de armazenamento de dados e regras de negócio da aplicação \textit{Web}. Com isto, é sabido que o serviço rodando no lado cliente serve como interface direta com o utilizador. Consequentemente, sistemas mal estruturados e incompletos podem fornecer ao usuário uma péssima primeira impressão. Portanto, é responsabilidade do desenvolvedor garantir que erros não ocorram, tais como: Ausência de retorno visual das interações; Tratamento correto de dados e validações; Aparência da aplicação e incompatibilidade entre dispositivos. Em outras palavras, é importante evitar erros que impactam o funcionamento do sistema e geram desconforto para a usabilidade. Neste contexto, este trabalho apresenta boas práticas de desenvolvimento de serviços \textit{Web} e explora técnicas de otimização da experiência do usuário ao navegar através de aplicações \textit{Web}. São utilizados padrões definidos por entidades reguladoras (W3C) e grandes grupos atuantes (WHATWG).

\vspace{\onelineskip}
 \noindent
 \textbf{Palavras-chave}: Aplicações \textit{Web}; Desenvolvimento; Otimização.