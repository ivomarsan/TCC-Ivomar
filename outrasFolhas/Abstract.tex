\vfill
\begin{center}
{\textbf{ABSTRACT}\\}
\end{center}

\noindent

Nowadays, it’s noticeable the growing use and dissemination of existing services on the World Wide Web, also known by the acronym WWW or simply Web, which is a system of interconnected documents and that are applied on the Internet. Its architecture is divided mainly into two interconnected parts, called client and server. In this model, the responsibilities are shared and guarantee the execution of services and systems of networks. The server side, as the name explicitly, is responsible for serving its customers, making the service available to respond to a customer's demand. On the other hand, the client requests the service through a browser, initiating the interaction from a user request. In this way, the user can perform dynamic content requests with the server, through several communication protocols, being HTTP (Hypertext Transfer Protocol) the most common. In addition, the technologies employed for the perfect functioning of a Web application are also subdivided. It is common to note that, generally, every visual interaction is located on the client side. Even in the server side, it locates the service of data storage and business rules of the Web application.  Thus, it is known that the service running on the client side serves as a direct interface with the user. Consequently, poorly structured and incomplete systems can provide the user a poor first impression. Therefore, it is a developer’s responsibility ensure that errors do not occur, such as: absence of visual feedback of interactions; Correct treatment of data and validations; Appearance of the application and incompatibility between devices. In other words, it is important to avoid errors that impact the functioning of the system and generate discomfort for usability. In this context, this paper presents a set of good Web service development practices, exploring user experience optimization techniques during the use of Web applications. Standards defined by regulators (W3C) and large working groups (WHATWG) are used. 
 
 \vspace{\onelineskip}
    
 \noindent
 \textbf{Keywords}: Web Applications; Development; Optimization.
