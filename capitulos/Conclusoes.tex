\chapter{CONCLUSÃO}
\label{Conclusao}

Com a rápida evolução de soluções \textit{Web}, surgem também diversas formas de desenvolvimento. No entanto, é importante definir quais são as melhores ferramentas e estratégias para alcançar os objetivos de uma aplicação \textit{Web} de qualidade. Levando isso em consideração, este trabalho apresentou um conjunto de técnicas amplamente utilizadas para a otimização de aplicações \textit{Web}, exemplificando através de \textit{scripts} didáticos e de um sistema real em funcionamento.

A partir dos resultados, conclui-se neste trabalho que, a aplicação de boas práticas de programação durante o período de desenvolvimento, influenciam diretamente na performance da aplicação. Como é possível perceber nesta proposta, as otimizações descritas, quando implementadas corretamente, produzem significativos resultados. Esta percepção se faz possível ao utilizar técnicas de potencialização da aplicação focadas em fluidez e experiência do usuário.

\section{Trabalhos Futuros}
\label{TrabalhosFuturos}

Podem ser apresentados, na sequência deste trabalho, a introdução de boas práticas para o desenvolvimento de aplicações \textit{Web} utilizando novos conceitos referente às tecnologias citadas neste trabalho. São exemplos de trabalhos futuros: Técnicas de otimização de código e obtenção de maior desempenho na execução, além da abordagem de novos recursos para as futuras versões da HTML, da CSS e do JS; Adaptação na estrutura e arquitetura da aplicação para se obter o máximo de integração com as novas versões dos protocolos para transferência de arquivos, que causam impacto direto no desempenho das aplicações; Abordagem e inclusão de modelos e conceitos de aplicações \textit{Web}, tais como \textit{Single Page Application} (SPA) e \textit{Progressive Web Apps} (PWA); Integração com novas tecnologias, como \textit{Web Assembly}.