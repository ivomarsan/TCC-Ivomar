\chapter{INTRODUÇÃO}
\label{Introducao}

Com os avanços nas tecnologias de rede e comunicações, a \textit{Web} se tornou o meio mais utilizado para a distribuição e acesso de informações, inclusive sobre a própria \textit{Web}. Neste processo, tecnologias e linguagens foram projetadas para suprir essa demanda mais crescente de soluções na Internet. Este fato está ligado diretamente às necessidades em tornar os ambientes computacionais, cada vez mais, conectados, interativos e acessíveis.

Ao se pensar no desenvolvimento de um projeto \textit{Web}, é extremamente importante ter uma base sólida de conhecimento sobre as tecnologias disponíveis, pois isto minimiza os problemas inerentes ao processo de criação de uma aplicação para Internet. Portanto, o primeiro passo que deve-se realizar para a criação de uma boa estrutura de \textit{software} é entender, de forma detalhada, as principais tecnologias, respeitando as exigências fundamentais de uma arquitetura confiável para \textit{Web}, sendo esta segura, de fácil manutenção, de alto desempenho, etc. Sendo assim, é impossível desenvolver uma aplicação \textit{Web} de qualidade e escalável sem, antes, compreender e planejar os principais desafios enfrentados durante a fase de desenvolvimento da solução.

Tendo em vista tais desafios, este trabalho aborda um conjunto de técnicas aprovadas e consolidadas pelas comunidades de desenvolvimento \textit{Web}. Estas técnicas têm como objetivo otimizar o desempenho das aplicações \textit{Web}, através da utilização das boas práticas de desenvolvimento no \textit{Front-end}.

No entanto, dominar uma tecnologia não é simplesmente saber como desenvolver uma aplicação, mas saber como esta aplicação se comporta sempre que um recurso é requisitado. Por isso, o guia proposto aborda os conceitos fundamentais de como funciona uma aplicação Web, revisando alguns métodos do protocolo HTTP e técnicas de programação assíncrona com JavaScript.

\section{Motivação}
\label{Motivacao}

É sabido que, considerando a grande quantidade de informação disponível sobre as tecnologias criadas para a \textit{Web}, como as linguagens HTML, CSS e JavaScript, bem como a grande quantidade de bibliotecas e \textit{frameworks} que oferecem suporte ao desenvolvimento de aplicações usando essas tecnologias, o aprendizado sobre como utilizá-las de forma correta nem sempre é uma tarefa fácil para desenvolvedores.

Para minimizar este problema de aprendizagem, este trabalho apresenta, de forma simples e objetiva, um guia sobre como utilizar essas tecnologias em qualquer aplicação Web para o \textit{front-end}.

O uso correto destas tecnologias torna o desenvolvimento de aplicações \textit{Web} mais fácil e produtivo, do ponto de vista da qualidade das aplicações desenvolvidas, devido à sua padronização de código. A padronização também possibilita uma melhor manutenção de código.
%
%
%
%
%
%
%
%
%
\section{Objetivos}
\label{Objetivos}

\subsection{Objetivo Geral}
\label{ObjetivoGeral}

Este trabalho tem como objetivo geral apresentar, de forma sistemática, um guia sobre como utilizar as principais técnicas disponíveis na atualidade. A ideia é apresentar a proposta para atingir o máximo de performance no desenvolvimento de qualquer aplicação \textit{Web} no \textit{Front-end}.
%
%
%
\subsection{Objetivos Específicos}
\label{ObjetivosEspecificos}

Em particular, o presente trabalho propõe a implementação de técnicas e boas práticas para obter otimizações, tais como a diminuição do \textit{payload} e o aumento da velocidade de renderização em páginas \textit{Web}, tornando a aplicação mais fluida e assim proporcionando uma melhor experiência para o usuário.
%
%
%
%
%
%
%
%
%
\section{Organização do Trabalho}
\label{Organizacao}

Este trabalho está organizado como se segue. No Capítulo~\ref{FundamentacaoTeorica} é feita a fundamentação teórica, detalhando as tecnologias que são necessárias para o entendimento do restante do trabalho, mostrando o histórico de cada tecnologia, bem como sua evolução e funcionamento. No Capítulo~\ref{Proposta} é detalhada a proposta, visando potencializar o desempenho de aplicações \textit{Web} modernas, através do uso de boas práticas de desenvolvimento, além de explanar as técnicas e introduzir ferramentas complementares para auxiliar no desenvolvimento utilizando \textit{code-pattern}. O Capítulo~\ref{Resultados} descreve os resultados dos experimentos realizados a partir da abordagem proposta, além de detalhar as técnicas utilizadas para a obtenção dos resultados após a aplicação das boas práticas mencionadas no trabalho para a otimização do desempenho, comprovando assim a efetividade das técnicas no decorrer dos testes. Por fim, no Capítulo~\ref{Conclusao} é realizada a conclusão deste trabalho, dando direcionamentos para trabalhos futuros.